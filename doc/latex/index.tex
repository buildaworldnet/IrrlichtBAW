\hypertarget{index_intro}{}\section{Introduction}\label{index_intro}
Welcome to the irr\+X\+ML A\+PI documentation. Here you\textquotesingle{}ll find any information you\textquotesingle{}ll need to develop applications with irr\+X\+ML. If you look for a tutorial on how to start, take a look at the \hyperlink{index_irrxmlexample}{Example}, at the homepage of irr\+X\+ML at \href{http://www.ambiera.com/irrxml/}{\tt www.\+ambiera.\+com/irrxml/} or into the S\+DK in the directory example.

irr\+X\+ML is intended to be a high speed and easy-\/to-\/use X\+ML Parser for C++, and this documentation is an important part of it. If you have any questions or suggestions, just send a email to the author of the engine, Nikolaus Gebhardt (niko (at) irrlicht3d.\+org). For more informations about this parser, see \hyperlink{index_history}{History}.\hypertarget{index_features}{}\section{Features}\label{index_features}
irr\+X\+ML provides forward-\/only, read-\/only access to a stream of non validated X\+ML data. It was fully implemented by Nikolaus Gebhardt. Its current features are\+:


\begin{DoxyItemize}
\item It it fast as lighting and has very low memory usage. It was developed with the intention of being used in 3D games, as it already has been.
\item irr\+X\+ML is very small\+: It only consists of 60 KB of code and can be added easily to your existing project.
\item Of course, it is platform independent and works with lots of compilers.
\item It is able to parse A\+S\+C\+II, U\+T\+F-\/8, U\+T\+F-\/16 and U\+T\+F-\/32 text files, both in little and big endian format.
\item Independent of the input file format, the parser can return all strings in A\+S\+C\+II, U\+T\+F-\/8, U\+T\+F-\/16 and U\+T\+F-\/32 format.
\item With its optional file access abstraction it has the advantage that it can read not only from files but from any type of data (memory, network, ...). For example when used with the Irrlicht Engine, it directly reads from compressed .zip files.
\item Just like the Irrlicht Engine for which it was originally created, it is extremely easy to use.
\item It has no external dependencies, it does not even need the S\+TL.
\end{DoxyItemize}

Although irr\+X\+ML has some strenghts, it currently also has the following limitations\+:


\begin{DoxyItemize}
\item The input xml file is not validated and assumed to be correct.
\end{DoxyItemize}\hypertarget{index_irrxmlexample}{}\section{Example}\label{index_irrxmlexample}
The following code demonstrates the basic usage of irr\+X\+ML. A simple xml file like this is parsed\+: 
\begin{DoxyCode}
<?xml version=\textcolor{stringliteral}{"1.0"}?>
<config>
 <!-- This is a config file \textcolor{keywordflow}{for} the mesh viewer -->
 <model file=\textcolor{stringliteral}{"dwarf.dea"} />
 <messageText caption=\textcolor{stringliteral}{"Irrlicht Engine Mesh Viewer"}>
 Welcome to the Mesh Viewer of the &quot;Irrlicht Engine&quot;.
 </messageText>
</config>
\end{DoxyCode}


The code for parsing this file would look like this\+: 
\begin{DoxyCode}
\textcolor{preprocessor}{#include <\hyperlink{irrXML_8h}{irrXML.h}>}
\textcolor{keyword}{using namespace }\hyperlink{namespaceirr}{irr}; \textcolor{comment}{// irrXML is located in the namespace irr::io}
\textcolor{keyword}{using namespace }io;

\textcolor{preprocessor}{#include <string>} \textcolor{comment}{// we use STL strings to store data in this example}

\textcolor{keywordtype}{void} main()
\{
 \textcolor{comment}{// create the reader using one of the factory functions}

 \hyperlink{namespaceirr_1_1io_a1628edbb9d5d53f18c82d2a92b0ad27e}{IrrXMLReader}* xml = \hyperlink{namespaceirr_1_1io_a581f4d4648398759c61266d63d7106b1}{createIrrXMLReader}(\textcolor{stringliteral}{"config.xml"});

 \textcolor{comment}{// strings for storing the data we want to get out of the file}
 std::string modelFile;
 std::string messageText;
 std::string caption;

 \textcolor{comment}{// parse the file until end reached}

 \textcolor{keywordflow}{while}(xml && xml->read())
 \{
     \textcolor{keywordflow}{switch}(xml->getNodeType())
     \{
     \textcolor{keywordflow}{case} \hyperlink{namespaceirr_1_1io_a86a02676c9cbb822e04d60c81b4f33eda0edf973f8ca0f6097f69369539d432a4}{EXN\_TEXT}:
         \textcolor{comment}{// in this xml file, the only text which occurs is the messageText}
         messageText = xml->getNodeData();
         \textcolor{keywordflow}{break};
     \textcolor{keywordflow}{case} \hyperlink{namespaceirr_1_1io_a86a02676c9cbb822e04d60c81b4f33eda9df4f5baccc23a0ad1f6fa64d8de2fc0}{EXN\_ELEMENT}:
         \{
             \textcolor{keywordflow}{if} (!strcmp(\textcolor{stringliteral}{"model"}, xml->getNodeName()))
                 modelFile = xml->getAttributeValue(\textcolor{stringliteral}{"file"});
             \textcolor{keywordflow}{else}
             \textcolor{keywordflow}{if} (!strcmp(\textcolor{stringliteral}{"messageText"}, xml->getNodeName()))
                 caption = xml->getAttributeValue(\textcolor{stringliteral}{"caption"});
         \}
         \textcolor{keywordflow}{break};
     \}
 \}

 \textcolor{comment}{// delete the xml parser after usage}
 \textcolor{keyword}{delete} xml;
\}
\end{DoxyCode}
\hypertarget{index_howto}{}\section{How to use}\label{index_howto}
Simply add the source files in the /src directory of irr\+X\+ML to your project. Done.\hypertarget{index_license}{}\section{License}\label{index_license}
The irr\+X\+ML license is based on the zlib license. Basicly, this means you can do with irr\+X\+ML whatever you want\+:

Copyright (C) 2002-\/2012 Nikolaus Gebhardt

This software is provided \textquotesingle{}as-\/is\textquotesingle{}, without any express or implied warranty. In no event will the authors be held liable for any damages arising from the use of this software.

Permission is granted to anyone to use this software for any purpose, including commercial applications, and to alter it and redistribute it freely, subject to the following restrictions\+:


\begin{DoxyEnumerate}
\item The origin of this software must not be misrepresented; you must not claim that you wrote the original software. If you use this software in a product, an acknowledgment in the product documentation would be appreciated but is not required.
\item Altered source versions must be plainly marked as such, and must not be misrepresented as being the original software.
\item This notice may not be removed or altered from any source distribution.
\end{DoxyEnumerate}\hypertarget{index_history}{}\section{History}\label{index_history}
As lots of references in this documentation and the source show, this xml parser has originally been a part of the \href{http://irrlicht.sourceforge.net}{\tt Irrlicht Engine}. But because the parser has become very useful with the latest release, people asked for a separate version of it, to be able to use it in non Irrlicht projects. With irr\+X\+ML 1.\+0, this has now been done. 